\documentclass{beamer}

\usetheme{Antibes}
\usepackage{verbatim}
\usecolortheme[RGB={120,0,0}]{structure}
\setbeamertemplate{blocks}[rounded][shadow=true]
\usepackage[utf8]{inputenc}

\beamertemplateballitem

\begin{document}

\title{UKP5: a New Algorithm for the Unbounded Knapsack Problem}
\author{Henrique Becker\footnote{From the Universidade Federal do Rio Grande do Sul\label{ft:ufrgs}} and Luciana S. Buriol\textsuperscript{\ref{ft:ufrgs}}\\
15th International Symposium on Experimental Algorithms\\
SEA 2016
}
\logo{\includegraphics[scale=0.3]{inf}}
\date{Wednesday, June 8}

\frame{\titlepage}

\frame{\frametitle{Outline}\tableofcontents}

\section{Introduction}
\frame{\frametitle{What is UKP?}
\begin{itemize}
\item UKP is an acronym for Unbounded Knapsack Problem
\item Very similar to BKP or 0-1 KP
\begin{itemize}
\item But with an "unbounded" quantity of each item available
\end{itemize}
\item Subproblem of the column generation technique (often used to solve CSP/BPP)
\end{itemize}
}

\frame{\frametitle{UKP Model}
\begin{align}
  maximize &\sum_{i=1}^n p_i x_i\label{eq:objfun}\\
subject~to &\sum_{i=1}^n w_i x_i \leq c\label{eq:capcons}\\
            &x_i \in \mathbb{N}_0\label{eq:x_integer}
\end{align}
}

\section{Prior Work}
\frame{\frametitle{Methods on Solving UKP}
\begin{beamerboxesrounded}[shadow=true]{Dynamic Programming (DP)}
Strong correlation between 'n' and 'c' and the time to solve.\\
Ex.: 
\end{beamerboxesrounded}	
\begin{beamerboxesrounded}[shadow=true]{Branch and Bound (B\&B)}
Strong correlation between instance's item distribution and time to solve.\\
Ex.:
\end{beamerboxesrounded}	
}

\frame{\frametitle{PYAsUKP/EDUK2 (State-of-art)}
\begin{beamerboxesrounded}[shadow=true]{REFERENCE}
``EDUK [...] seems to be the most efficient dynamic programming based method available at the moment.''
\end{beamerboxesrounded}
\begin{itemize}
\item EDUK2 is the algorithm, PYAsUKP is the only known implementation (OCaml). %(As the algorithm uses many functional concepts the authors found that it would be so much easier to implement the algorithm on a functional language. Even if c++ is the de facto language for scientific computing.)
\item \textbf{Hybrid Approach}: Tries to solve by B\&B, if fails to solve quickly then defaults to DP. %(Only hybrid method known.)
\end{itemize}
}

\section{UKP5}
\frame{\frametitle{UKP5}
\begin{itemize}
\item DP algorithm based on [Garfinkel]
\end{itemize}
}

\frame{\frametitle{UKP5 Code}
% PUT CODE HERE
}

\frame{\frametitle{GG66}
\begin{itemize}
\item "Ordered Step-Off": algorithm of Gilmore and Gomore (1966).
\item Found after sending this paper to the conference.%(We notified the conference chair, but the paper was already submited for printing.)
\item Basically the same algorithm than UKP5.%(only uses the own vector for storing the profit of the best solution, instead of one single variable)
\end{itemize}
}

\frame{\frametitle{GG66 Code}
% PUT CODE HERE
}

\begin{comment}

> Experiments and Results

The instances
	A set of 4540 "hard" instances proposed in [PYAsUKP]. (Hard is a little complex to define here. Some formulae were proven to create instances that are hard to solve by B\&B (hard to reach upper bound); some are guaranteed to be resistent to methods that allow to reduce the number of items or the capacity of the instance).
	Used the same tool and similar parameters to generate them. (The tool is the own pyasukpt, that isn't only an implementation of the EDUK2 but also a implementation of many instance generators.)
	Five different classes of instances. (We adopted the same five classes used on PYAsUKP, and only augmented one of the classes because with the hardwere evolution it had become too easy to solve.)

Times table

Times graph

> Analysis and Final Remarks

#Solution dominance
#	Applying dominance is discarding items that can't be on the optimal solution.
#	DP algorithms as UKP5/GG66 can do that with almost no overhead while executing.

Hard vs Easy \& DP vs B\&B
	PYAsUKP is mainly a DP method. (the cases where the B\&B don't solve the instance instantly, it don't affect the PYAsUKP time very much)
	The benchmark proposed at PYAsUKP's paper focused hard-to-solve-by-B\&B instances. (Instances with similar profit-to-weight ratio; them most efficient items being the ones with the biggest weight; etc..)
	PYAsUKP compared its results against the ones from a B\&B method, and against no DP method. (the method was MTU2)

Final remarks
	The algorithm isn't novel, but is "faster" than the "state-of-art" (At least for the instances proposed by authors of the state-of-art themselves)
	Except by PYAsUKP, DP algorithms were abandoned with almost no empiric evidence (No paper presenting a DP algorithm compared computational results with another DP algorithm. And no paper presenting a B\&B algorithm compared to DP (only to other B\&B).)
	A clear definition of the usefulness of each approach is necessary (We need to know for what instance sizes and other instance characteristics each approach is better, and we need to base this on solid empiric evidence.)

Future Works
	A survey on the UKP
		Improve the benchmark dataset (include very big and random instances; include instances generated by the CPS/BPP column generation; create instances so hard that will make UKP5 take more than 1000s)
		Test old algorithms that are provided with no computational results (many were already implemented after we send the paper)
		Point that UKP5/GG66 are the same algorithm, and apologize (avoid that the misconception created by our article become long-lived)

References

Questions?


\subsection{Problems}
\frame{\frametitle{bla}
\begin{beamerboxesrounded}[shadow=true]{using boxes here...}
\begin{itemize}
\item 
\end{itemize}
\end{beamerboxesrounded}	
}

\section{other section}
\frame{\frametitle{...}
\begin{beamerboxesrounded}[shadow=true]{...}
\begin{itemize}
\item ...
\end{itemize}
\end{beamerboxesrounded}	
}

\section{Conclusions and Future Work}
\frame{
\begin{beamerboxesrounded}[shadow=true]{Conclusions}
\begin{itemize}
\item In this paper ...
\begin{itemize}
\item ...
\item ...
\end{itemize}
\item ...
\end{itemize}. 
\end{beamerboxesrounded}
}

\frame{
\frametitle{Acknowledgments}
inclua sua imagem aqui

\includegraphics[scale=0.55]{inf} 
}
\end{comment}

\end{document}

