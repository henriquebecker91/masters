Uma revisão dos algoritmos e conjuntos de instâncias presentes na literatura do Problema da Mochila com Repetições (PMR) é apresentada.
Os algoritmos e conjuntos de instâncias usados são brevemente descritos nesse trabalho, a fim de que o leitor tenha base para entender as discussões.
Algumas propriedades bem conhecidas e específicas do PMR, como dominância e periodicidade, são explicadas com detalhes.
O PRM é também superficialmente estudado no contexto de problemas de avaliação gerados pela abordagem de geração de colunas aplicada na relaxação contínua do \emph{Bin Packing Problem (BPP)} e o \emph{Cutting Stock Problem (CSP)}.
Múltiplos experimentos computacionais e comparações são realizadas.
Para os conjuntos de instâncias artificiais mais recentes da literatura, um simples algoritmo de programação dinâmica, e uma variante do mesmo, parecem superar o desempeho do resto dos algoritmos, incluindo aquele que era estado-da-arte.
O modo que relações de dominância é aplicado por esses algoritmos de programação dinâmica têm algumas implicações para as relações de dominância previamente estudadas na literatura.
O autor defende a tese de que a escolha dos conjuntos de instâncias artificiais definiu o que foi considerado o melhor algoritmo nos trabalhos anteriores.
Todos os códigos e conjuntos de instâncias usados foram disponibilizados pelo autor.
 
