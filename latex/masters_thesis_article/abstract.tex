An empirical analysis of exact algorithms for the unbounded knapsack problem was carried out.
The experiments included seven algorithms from the literature and more than ten thousand instances.
The terminating step-off, a dynamic programming algorithm from 1966, was found to have the lowest mean time to solve the most recent benchmark from the literature.
The terminating step-off seems to partially and implicitly explore threshold and collective dominance, which are properties of the unbounded knapsack problem first discussed in 1998.
A class of instances that favors the branch-and-bound approach over the dynamic programming approach without displaying high amounts of simple, multiple and collective dominance is presented.
The pricing subproblems from solving hard cutting stock problems with column generation seem to be best solved with dynamic programming as the branch-and-bound algorithms tested displayed worst case times.
The definition of which instances are of interest defines which algorithm is considered the best.
All algorithm and instance generation code used is available online.

